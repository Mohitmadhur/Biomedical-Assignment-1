\documentclass[12pt]{article}


\begin{document}
\tableofcontents


\section{CAST SAW}






\subsection{INTRODUCTION}
A cast is often used for the treatment of a broken bone, post-surgical recovery, or other ailments that require immobilization. Casts are made of one of two materials: plaster or fiberglass. Once it is time for the cast to come off, a cast saw is used to remove it. Learn about how a cast saw works, how safe it is, and what you can do to make this a less frightening experience
The patient's skin frequently comes into contact with the cast saw blade without cutting although it can cause lacerations when used over bony prominences.[3] The design enables the saw to cut rigid materials such as plaster or fiberglass while soft tissues such as skin move back and forth with the blade, dissipating the shear forces, preventing injury] A general technique in the use of cast saw often involves a demonstration before actually cutting the cast.

\subsection{PROS AND CONS}
PROS OF USING CAST SAW
Cast saws are very safe, but they should only be used by personnel who have been trained in their proper use and how to avoid problems. Improper use of a cast saw, or use of a cast saw that has worn blades, can cause problems. Cast saws are safe, but there are possible complications of their use that can occur. Although it is possible in some cases to sustain a small skin injury or burn from a cast saw, with proper use, these injuries are not common, and there are precautions that can be taken to minimize this risk
CONS OF USING CAST SAW
There are a few problems that can occur with the use of a cast saw, and it is important that the device is used by someone knowledgeable with proper cast saw techniques. While a healthcare provider should know proper cast saw techniques, many cast techs, physician assistants, and medical assistants are also qualified in using this device.
The most common reasons patients had complications from the use of a cast saw, according to one study, were worn-out blades, insufficient cast padding, or improper training and experience. The rate of injury from a cast saw is right around 1%; therefore, the risk is small, but not nonexistent.3 The most common problems include the following
Burns: Skin burns are the most common problem that can occur when removing a cast with a cast saw. Because of the vibration of the cast saw blade, high temperatures can result from the friction of the blade against the cast material. 
Cuts: Small skin lacerations are uncommon, but can occur. The teeth of the saw blade can be sharp enough to scratch the skin. If ample padding is under the hard cast material, a skin laceration is less likely.






\subsection{WAY FORWARD}
Many patients, especially younger children, are frightened of cast saws, but there are measures that can be taken to make the experience less traumatic.
Explain to kids what is happening. Don't let the healthcare provider or cast tech rush in and start removing the cast without showing the patient the equipment and how it works. Fear of the unknown is usually much worse than fear of the saw.
Show the patient that the saw will not cut the skin. Skin lacerations are the most common fear, and demonstrating that the saw will not cut your skin can help: your healthcare provider or technician may press the blade of the running cast saw against their hand to demonstrate that it's safe.
Bring headphones. A cast saw can be noisy, and often the noise is more upsetting than the actual feeling of the saw. Earmuffs, headphones, or a noise-canceling device can help. Often kids will enjoy listening to music while the cast is being removed.
Even with these steps, some patients are still upset and frightened. Taking time and addressing the patient's concerns can help. Unfortunately, some kids are too young to understand, and that's where a promise of an ice cream treat may be the only thing that helps get them through!



\subsection{CONCLUSION}
Casts are commonly used for treatment in orthopedics. Many people, at some point or another in their life, will have a cast placed for treatment of a broken bone or immobilization after surgery. While removal of the cast can provoke anxiety, it is actually a very safe procedure. A skilled cast technician will take steps to ensure that your cast is removed safely and without complications.


\section{MRI}



\subsection{INTRODUCTION}
MRI stands for Magnetic resonance imaging (MRI).It is a medical imaging technique that uses a magnetic field and computer-generated radio waves to create detailed images of the organs and tissues in your body.
Most MRI machines are large, tube-shaped magnets. When you lie inside an MRI machine, the magnetic field temporarily realigns water molecules in your body. Radio waves cause these aligned atoms to produce faint signals, which are used to create cross-sectional MRI images — like slices in a loaf of bread.
The MRI machine can also produce 3D images that can be viewed from different angles. It is often used for disease detection, diagnosis, and treatment monitoring. It is based on sophisticated technology that excites and detects the change in the direction of the rotational axis of protons found in the water that makes up living tissues.

\subsection{HOW IT WORKS}
MRIs employ powerful magnets which produce a strong magnetic field that forces protons in the body to align with that field. When a radiofrequency current is then pulsed through the patient, the protons are stimulated, and spin out of equilibrium, straining against the pull of the magnetic field. When the radiofrequency field is turned off, the MRI sensors are able to detect the energy released as the protons realign with the magnetic field. The time it takes for the protons to realign with the magnetic field, as well as the amount of energy released, changes depending on the environment and the chemical nature of the molecules. Physicians are able to tell the difference between various types of tissues based on these magnetic properties.
\subsection{WHY IT IS USED}
MRI scanners are particularly well suited to image the non-bony parts or soft tissues of the body. They differ from computed tomography (CT), in that they do not use the damaging ionizing radiation of x-rays. The brain, spinal cord and nerves, as well as muscles, ligaments, and tendons are seen much more clearly with MRI than with regular x-rays and CT; for this reason MRI is often used to image knee and shoulder injuries.
In the brain, MRI can differentiate between white matter and grey matter and can also be used to diagnose aneurysms and tumors. Because MRI does not use x-rays or other radiation, it is the imaging modality of choice when frequent imaging is required for diagnosis or therapy, especially in the brain. However, MRI is more expensive than x-ray imaging or CT scanning.
One kind of specialized MRI is functional Magnetic Resonance Imaging (fMRI.) This is used to observe brain structures and determine which areas of the brain “activate” (consume more oxygen) during various cognitive tasks. It is used to advance the understanding of brain organization and offers a potential new standard for assessing neurological status and neurosurgical risk.

\subsection{RISKS}
Although MRI does not emit the ionizing radiation that is found in x-ray and CT imaging, it does employ a strong magnetic field. The magnetic field extends beyond the machine and exerts very powerful forces on objects of iron, some steels, and other magnetizable objects; it is strong enough to fling a wheelchair across the room. Patients should notify their physicians of any form of medical or implant prior to an MR scan.
When having any MRI Scan the following people should be taken into consideration
People with implants, particularly those containing iron, — pacemakers, vagus nerve stimulators, implantable cardioverter- defibrillators, loop recorders, insulin pumps, cochlear implants, deep brain stimulators, and capsules from capsule endoscopy should not enter an MRI machine.
Noise—loud noise commonly referred to as clicking and beeping, as well as sound intensity up to 120 decibels in certain MR scanners, may require special ear protection
Nerve Stimulation—a twitching sensation sometimes results from the rapidly switched fields in the MRI.
Contrast agents—patients with severe renal failure who require dialysis may risk a rare but serious illness called nephrogenic systemic fibrosis that may be linked to the use of certain gadolinium-containing agents, such as gadodiamide and others. Although a causal link has not been established, current guidelines in the United States recommend that dialysis patients should only receive gadolinium agents when essential, and that dialysis should be performed as soon as possible after the scan to remove the agent from the body promptly.
Pregnancy—while no effects have been demonstrated on the fetus, it is recommended that MRI scans be avoided as a precaution especially in the first trimester of pregnancy when the fetus’ organs are being formed and contrast agents, if used, could enter the fetal bloodstream.


\subsection{WAY FORWARD}
Replacing Biopsies with Sound
Chronic liver disease and cirrhosis affect more than 5.5 million people in the United States. NIBIB-funded researchers have developed a method to turn sound waves into images of the liver, which provides a new non-invasive, pain-free approach to find tumors or tissue damaged by liver disease. The Magnetic Resonance Elastography (MRE) device is placed over the liver of the patient before he enters the MRI machine. It then pulses sound waves through the liver, which the MRI is able to detect and use to determine the density and health of the liver tissue. This technique is safer and more comfortable for the patient as well as being less expensive than a traditional biopsy. Since MRE is able to recognize very slight differences in tissue density, there is the potential that it could also be used to detect cancer.
New MRI just for Kids
MRI is potentially one of the best imaging modalities for children since unlike CT, it does not have any ionizing radiation that could potentially be harmful. However, one of the most difficult challenges that MRI technicians face is obtaining a clear image, especially when the patient is a child or has some kind of ailment that prevents them from staying still for extended periods of time.
Determining the aggressiveness of a tumor
Traditional MRI, unlike PET or SPECT, cannot measure metabolic rates. However, researchers funded by NIBIB have discovered a way to inject specialized compounds (hyperpolarized carbon 13) into prostate cancer patients to measure the metabolic rate of a tumor. This information can provide a fast and accurate picture of the tumor’s aggressiveness. Monitoring disease progression can improve risk prediction, which is critical for prostate cancer patients who often adopt a wait and watch approach.




\section{SLT LASER}




\subsection{INTRODUCTION}
SLT is a laser that treats the drain directly to help increase the outflow of fluid. It treats specific cells”selectively,” leaving the trabecular meshwork intact. For this reason, SLT may be safely repeated. The surgical process involves numbing the eye with topical eye drops so that you will not feel the laser treatment. In SLT, laser treatment is applied to the drain of your eye in order to open it up and let fluid out, lowering the eye pressure and saving your sight. SLT treatment takes only a few minutes, is performed in the office, is safe, and effectively lowers eye pressure in most people. The treatment is approved by the Food and Drug Administration (FDA) for treating glaucoma and is covered by essentially all insurance plans. After the procedure anti-inflammatory drops are used.


\subsection{CANDIDATES FOR SLT}
Patients who have primary or secondary open-angle glaucoma (the drainage system in the front part of the eye is open) and are in need of lowering of their intraocular pressure (IOP) are eligible for the procedure. Your eye doctor will make the final determination if you are a candidate.

\subsection{HOW TO USE}
Laser energy is applied to the drainage tissue in the eye. This starts a chemical and biological change in the tissue that results in better drainage of fluid through the drain and out of the eye. This eventually results in lowering of IOP. It may take 1-3 months for the results to appear
Laser energy is applied to the trabecular meshwork. The trabecular meshwork is located in the front of the eye and is the natural drain for fluid. SLT stimulates the trabecular meshwork to increase the amount of fluid drained from within the eye, which lowers eye pressure. It may take up to 2 months for the laser to take its full effect.


 
 \subsection{PROS AND CONS}
 On average, SLT can lower the eye pressure by about 20-30%. The laser is effective in lowering eye pressure in about 80% of patients. The laser effect wears off after several years, but the trabecular meshwork can be stimulated again with repeat laser. SLT does not affect the success rates or the eligibility of getting other medical or surgical treatments.
Compared to many other treatments for glaucoma, including eye drops, other laser treatments and surgery, SLT has few side effects. Inflammation is possible after the laser but it is not common. You may be prescribed anti-inflammatory drops on an individual basis. There is a small risk of temporary high eye pressures shortly following the laser treatment. This is usually controlled with glaucoma medications  One key aspect of SLT is a favorable side effect profile, even when compared with glaucoma medications. Post-operative inflammation is common but generally mild, and treated with observation or eye drops or an oral non-steroidal anti-inflammatory drug. There is an approximately 5% incidence of IOP elevation after laser, which can be managed by glaucoma medications and usually goes away after 24 hours.


 \subsection{CONCLUSION}
 \SLT is a laser treatment for open-angle glaucoma that lowers eye pressure. It can be used as initial treatment, instead of eye drop medications, or as additional treatment when medications do not adequately reduce the eye pressure. It is often effective but that effectiveness may wear off after some period of time. It can be repeated but the effect may be reduced with repeat treatment. SLT is not a cure for glaucoma but one of many tools to keep it under control.

 
\section{GAMMA KNIFE}




 \subsection{INTRODUCTION} 
 Gamma Knife® is a radiation therapy that uses computerized treatment planning software to help physicians locate and irradiate small targets within the head and brain with very high precision. The treatment delivers intense radiation doses to the target area while sparing surrounding tissue.
If you're scheduled for radiation therapy using Gamma Knife®, a treatment team consisting of a radiation oncologist, a medical physicist and a neurosurgeon will work together to provide your treatment. Safety is ensured by the medical physicist who tests the machine's mechanical functions and verifies that the imaging and treatment planning computers and software are correct and acceptable
A Gamma Knife typically contains 201 cobalt-60 sources of approximately 30 curies each (1.1 TBq), placed in a hemispheric array in a heavily shielded assembly. The device aims gamma radiation through a target point in the patient's brain. The patient wears a specialized helmet that is surgically fixed to the skull, so that the brain tumor remains stationary at the target point of the gamma rays. An ablative dose of radiation is thereby sent through the tumor in one treatment session, while surrounding brain tissues are relatively spare.
Gamma Knife® surgery is a treatment method that uses radiation and computer-guided planning to treat brain tumors, vascular malformations and other abnormalities in the brain. Despite its name, this procedure does not involve any incisions, not even a skin incision. The Gamma knife is actually a treatment that delivers beams of highly focused radiation. Some 192 "beamlets" of radiation converge and are precisely focused on the targeted area of brain, specifically in the shape of the tumor or lesion, while sparing the surrounding normal tissue.
Gamma Knife surgery is also known as stereotactic radiosurgery, Gamma Knife radiosurgery and Gamma Knife radiation.


 \subsection{CONDITION FOR GAMMA KNIFE SURGERY TREAT}
Gamma Knife surgery can treat several brain disorders, including:
Brain tumors (both cancer [malignant] and non-cancer [benign]): These tumors including brain metastases, pituitary adenomas, pinealomas, craniopharyngiomas, meningiomas, chordomas, chondrosarcomas and glial tumors.
Acoustic neuroma (vestibular schwannomas): This is a non-cancerous tumor that develops around the balance and hearing nerves that connect the inner ear with the brain.
Arteriovenous malformations (AVMs): This is an abnormal, snarled tangles of blood vessels.
Tremors: Tremors due to conditions including essential tremor or Parkinson’s disease.
Trigeminal neuralgia: This is an ongoing condition that affects a certain nerve in the face, causing extreme pain.
The Gamma Knife may be helpful if you have a brain lesion or tumor that can’t be reached by traditional surgery techniques or if you’re unable to undergo surgery due to your condition or age. It can also be combined with traditional surgery to prevent tumor regrowth. The Gamma Knife is also used for some conditions that require urgent treatment.

 
 \subsection{HOW DOES THE EQUIPMENT WORK}
 The Gamma Knife® utilizes a technique called stereotactic radiosurgery, which uses multiple beams of radiation converging in three dimensions to focus precisely on a small volume, such as a tumor, permitting intense doses of radiation to be delivered to that volume safely. Currently, the available models use advanced robotic technology to move the patient in submillimeter increments during treatment to focus radiation successfully on all parts of the target.
Treatments using Perfexion
Most treatments are given in a single session. Under local anesthesia, a special rigid head frame incorporating a three-dimensional coordinate system is attached to the patient's skull with four screws. Imaging studies, such as magnetic resonance imaging (MRI), computed tomography (CT), or angiography, are then obtained and the results are sent to the Gamma Knife®'s planning computer system.
Treatments using Icon
Icon treatments differ from Perfexion treatments in that:
Treatments are given in single or multiple sessions.
An external frame is not required.
6+Icon uses a combination of onboard stereotactic Cone Beam CT imaging and real-time infrared motion detection and management to achieve 0.15 mm treatment accuracy even without a rigid frame.
What studies have been done or are being done to show its effectiveness?
The number of peer-reviewed, published scientific articles documenting patient outcomes with Gamma Knife® far outweighs any other form of stereotactic radiosurgery. Gamma Knife® centers and universities have published more than 2,500 papers and have treated more than 1 million patients worldwide during the last 50 years. The fact that 75% of all published radiosurgical literature including most of the multicenter trials is based on the use of the Gamma Knife® is especially significant given that both Gamma Knife® and Linac systems were introduced in the same era.

How many patients have received Gamma Knife® treatment?
Over 1 million patients have been treated with Leksell Gamma Knife® and about 80,000 patients are treated every year.
What does the patient feel during the Gamma Knife® treatment?
There might be mild pain from administration of the local anesthetic used during placement of the head frame (similar to the sensation of having blood drawn). Patients have reported that they feel a pressure sensation when the frame is applied, but not pain.

What can a patient expect after Gamma Knife® treatment?
After the treatment session is finished, the head frame is removed. Sometimes there is a little bleeding from where the pins contact the patient’s head. Pressure is applied to stop the bleeding and Bandaids may be used to keep the pin sites clean. It is usually recommended that the patient refrain from physical activity over the next 18 to 24 hours.
How quickly will the Gamma Knife® treatment work?
The effects of Gamma Knife radiosurgery occur over a period of time that can range from several weeks to several years, depending on the condition being treated.


\subsection{CONCLUSION}
Gamma Knife therapy, like all radiosurgery, uses doses of radiation to kill cancer cells and shrink tumors, delivered precisely to avoid damaging healthy brain tissue. Gamma Knife radiosurgery is able to accurately focus many beams of gamma radiation on one or more tumors. Each individual beam is of relatively low intensity, so the radiation has little effect on intervening brain tissue and is concentrated only at the tumor itself.




 
 
 \section{PULSE OXIMETER}


\subsection{INTRODUCTION}
Pulse oximetry is a noninvasive test that measures the oxygen saturation level of your blood.
It can rapidly detect even small changes in oxygen levels. These levels show how efficiently blood is carrying oxygen to the extremities furthest from your heart, including your arms and legs.
The pulse oximeter is a small, clip-like device. It attaches to a body part, most commonly to a finger.
Medical professionals often use them in critical care settings like emergency rooms or hospitals. Some doctors, such as pulmonologists, may use them in office settings. You can even use one at home.


\subsection{HOW TO USE}
Follow your health care provider’s recommendations about when and how often to check your oxygen levels.
Be aware that multiple factors can affect the accuracy of a pulse oximeter reading, such as poor circulation, skin pigmentation, skin thickness, skin temperature, current tobacco use, and use of fingernail polish. To get the best reading from a pulse oximeter:
Follow the manufacturer’s instructions for use.
When placing the oximeter on your finger, make sure your hand is warm, relaxed, and held below the level of the heart. Remove any fingernail polish on that finger.
Sit still and do not move the part of your body where the pulse oximeter is located.
Wait a few seconds until the reading stops changing and displays one steady number.
Write down your oxygen levels with the date and time of the reading so you can easily track changes and report these to your health care provider.
pneumonia

\subsection{PURPOSE}
The purpose of pulse oximetry is to see if your blood is well oxygenated.
Medical professionals may use pulse oximeters to monitor the health of people with conditions that affect blood oxygen levels, especially while they’re in the hospital.
These can include:
chronic obstructive pulmonary disease (COPD)
anemia
heart attack or heart failure
congenital heart disease
Doctors use pulse oximetry for a number of different reasons, including:
to assess how well a new lung medication is working
to evaluate whether someone needs help breathing
to evaluate how helpful a ventilator is
to monitor oxygen levels during or after surgical procedures that require sedation
to determine whether someone needs supplemental oxygen therapy
to determine how effective supplemental oxygen therapy is, especially when treatment is new
to assess someone’s ability to tolerate increased physical activity
to evaluate whether someone momentarily stops breathing while sleeping — like in cases of sleep apnea — during a sleep study

\subsection{CONCLUSION}
Pulse oximetry is a quick, noninvasive, and completely painless test. It comes with no risks, aside from potential skin irritation from the adhesive used in some types of probes.
However, it’s not as accurate as clinical blood gas measurements, especially for people with darker skin tones.




\end{document}